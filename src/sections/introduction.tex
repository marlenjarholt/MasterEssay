%!TEX root = ../master.tex

\section{Introduction}
This essay introduces Frog, a functional typed programming language for the OTTR framework. The goal of Frog is to improve some well-known software engineering design principles, such as separation of concerns(SoC) and do not repeat yourself(DRY). Since Frog is supposed to be a typed functional programming language, this essay introduces functional programming and its theory and type theory. Furthermore, OTTR is a framework or language for representing ontology modelling patterns as parameterised ontologies. OTTR makes it possible to make uses-defined abstraction to recurring modelling patterns expanding into RDF-graphs. Since Frog will become a part of the OTTR framework, we introduces the different semantic technologies that OTTR works on and a more in-depth explanation on OTTR and the theory behind OTTR. 

\subsection{Outline}

This essay has the following outline:
\begin{itemize}
    \item \textbf{Section 2: Functional programming} focuses on the background for functional programming. We are establishing and explaining the theory behind functional programming and type theory. 
    \item \textbf{Section 3: Semantic Technologies} introduces the fundamental technologies made for the Semantic Web. Moreover, this section also gives an introduction to OTTR.
    \item \textbf{Section 4: Frog} suggests how Frog, the functional typed language for the OTTR framework, should be built. In addition, we explain which features that we want Frog to have and why.
\end{itemize}