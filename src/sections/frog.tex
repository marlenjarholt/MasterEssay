%!TEX root = ../master.tex
\section{Frog}

\subsection{Motivation}
There are no ways change or mainpulate the terms (lists, liters etc) in OTTR. Imagen that you 
make a template for storing information about the weather, you have made sevral hundres instances. Say that one of the 
triples the is made from expanding the template to rdf states somethings temprature in farenheit but you need the temprature in celcius.
For now you how to change the term, meaning that you have to change the degrees manually (or using a scrip) for every OTTR instances.
The goal of Frog is to solve this problem with being a programming language written for OTTR where we are abel to change 
the terms using function. In the farenheit celcius exampel the solution with frog is then to make a function that can be used 
in the template body to convert the farenheit variable given in the head to celcius. 
\\ \\
Having a language like Frog that can make simple functions to mainpulate terms can do more than just convert terms, it can 
also make it possible to add new inffered information. E.g. if we have the template ex-t:Person from figure (SETT INN LINK HER)
and we want to add a tripel to the expanded template containing the information about wheter the person is an adult or not. This 
is information that can be inffered from the age variable, we can make a new ottr:Tripel with the person as the object, ex-r:isAdult as the 
predicate and the result of the function ex-f:ageToAdult 
\subsection{Plans for Frog}