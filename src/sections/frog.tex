%!TEX root = ../master.tex
\section{Frog}

\subsection{Motivation}
\label{motivation}
The use of OTTR templates proposes some benefits, such as better \emph{abstractions} \autocite[9]{SLKK_OTTR_2021} and \emph{uniform modelling} \autocite{Lupp_tliam}.  OTTR uses template patterns as an abstraction for low-level RDF triples and OWL axioms. Nonetheless, one may say that OTTR has the possibility to abstract even further. For instance, OTTR has no support for making URIs. Before using a template, one has to construct the URIs if they are none existing. Imagine that we have made a template that formally describes a person, with their social security number, name, and age. Using the social security number in the URI is natural since this is uniquely identifying. Furthermore, we want to use the template on different sources, such as a tabular document and a CVS document. The template user has to construct the URIs differently in the different sources before using the template. OTTR being able to create the URIs in the template body would then further improve OTTR's benefit of abstraction. Additionally, creating URIs in the template body benefits uniform modelling because we only have to change the URI structure in the template rather than in every source. The result of creating URI in the template body is that it reduces \emph{maintenance}.\lhk{Creating URIs in the template also helps with e.g. greater applicability of templates and allows methods for URI creation to be publishable together with the templates}

\para
\lhk{Perhaps put this section into an \emph{example}-environment} Conceptualize that we want to store information about the weather. We want to use several sources for the data, perhaps several different CSV files. Most of the files use Celcius as their temperature scale, but a few of the files store the temperatures in Fahrenheit. We, however, want to store all the temperature data in one standard temperature scale, namely Celcius. When converting Fahrenheit to Celcius, we can either do the calculations manually or use a script. After converting this data, we can then assemble the converted data into OTTR instances. However, this may be solved in the OTTR templates if it had been possible to do a simple operation on the Fahrenheit degrees. Doing the transformation manually from Fahrenheit to Celcius is time-consuming for big data sets. The second solution is still a solution, but it requires the template user to either know how to make a script for converting the degrees or use an exciting one. Using or making a script uniform modeling modelingay interfere with one of the benefits of using OTTR, namely \emph{separation of design and content}\autocite[9]{SLKK_OTTR_2021}. Separation of design and content in terms of OTTR means that OTTR separates between the design of the knowledge base, the templates, and the knowledge base's content, the instances. The ontology experts can manage templates, and domain experts can use familiar tools to create instances. The domain expert is not necessarily an expert on how to use scrips or how to make them. Ideally, converting from Fahrenheit to Celcius should be part of the design of the knowledge base.  

\para
To solve these issues, we propose a functional language that can make functions to manipulate the terms in OTTR, called Frog. If needed, instances in a template body can use a Frog function to manipulate one or several arguments, for instance,  making a URI from a social security number or converting Fahrenheit to Celcius. As previously mentioned, OTTR has a type system. Therefore, naturally, Frog will use the same type system, resulting in Frog being a typed language with parameter types and a return type.  The benefit of Frog being a typed system is that we can perform type checking in the templates. It is crucial to note that Frog can only manipulate terms and not change the RDF graph resulting from an expansion. Therefore, Frog will be a purely functional language. We propose that Frog will utilise the same list structure for its functions similar to Scheme and other LISP languages. In addition, we propose that Forg's syntax will be written in RDF.

\para
Having a language like Frog also makes it possible to add inferred data. For instance, if we have the template ex-t:Person from earlier and we want to add a tripel to the expanded template containing the information about whether the person is of type adult (ex-p:Adult) or child (ex-p:Child). We can then infer this information from the age variable. As a result, we can make a new rdf-o:type with the person as the subject and the result of the function ex-r:ageToLifeStage as the object. Where ex-r:ageToLifeStage is a Frog function that takes in an xsd:integer as an argument and returns a ottr:IRI. The ottr template below shows the potential use of the Frog function in the ex-t:Person.

\begin{lstlisting}[frame=single]
    
ex-t:Person [
    ottr:IRI ?person,
    xsd:integer ?age,
    ? List<ottr:IRI> ?fathers,
    ? List<ottr:IRI> ?mothers,
    ? List<ottr:IRI> ?ancestors,
    List<xsd:String> ?names
    ] :: {
    cross | ottr:Triple(?person, ex-r:hasFather, ++?fathers),
    cross | ottr:Triple(?person, ex-r:hasMother, ++?mothers),
    ottr:Triple(?person, ex-r:hasAge, ?age),
    cross | ottr:Triple(?person, ex-r:hasName, ++?names),
    ottr:Triple(?person, ex-r:ancestor, ?ancestors),
    rdf-o:Type(?person, (ex-f:ageToLifeStage ?age))
}.
\end{lstlisting}

The expansion the two following two instances

\begin{lstlisting}[frame=single]
ottr:Triple(_:b, ex-r:hasFather, ex-p:Roger) .
ex-t:Person(ex-p:Sebastian, 22 , (ex-p:Thommas, _:b), none, 
(ex-p:Thommas, ex-p:Roger), ("Sebastian"@no, "Bastian"@en)).

ex-t:person(ex-p:Nina, 6, none, none ,none, ("Nina"@no))
\end{lstlisting}

will then be
\begin{lstlisting}[frame=single, language=turtle]
ex-p:Sebastian ex-r:hasFather ex-p:Thommas, [ ex-r:hasFather ex-p:Roger] ; 
                ex-r:hasAge  "22"^^xsd:int ; 
                ex-r:hasName  "Sebastian"@no, "Bastian"@en ;
                ex-r:ancestor (ex-p:Thommas ex-p:Roger);
                a ex-p:Adult.

ex-p:Nina ex-r:hasAge  "6"^^xsd:int ; 
          ex-r:hasName  "Nina"@no ;
          a ex-p:Child.
\end{lstlisting}

\subsection{Plans for Frog}
Frog is\lhk{perhaps rephrase the text to be in future tense (i.e.~"Frog will/should be...") as Frog does not yet exist?} a high-order function programming language, only allowing pure functions. Furthermore, Frog utilises RDF for expressing its syntax. Frog's goal is to make it feasible\lhk{perhaps "possible" is a better word here} to manipulate the terms in the OTTR template body using Frog functions. In addition, Frog has two types of functions, named and anonymous functions. Having anonymous functions makes it possible to write functions straight into the OTTR template body. However, we specify the named functions in another file, and OTTR can then use these functions in the OTTR template body. Since Frog is a part of the OTTR framework, it applies the same type and term system as OTTR. Frog requires that we declare the return type of the function and the parameter types explicitly, meaning that Frog is a typed programming language. We further elaborate on the use of types in Frog in \autoref{types in frog}.


\subsubsection{Evaluation}
\label{frog_evaluation}
Frog utilises lazy evaluation to evaluate its expressions. We have chosen to use lazy evaluation to ensure that Frog does not perform unnecessary evaluations. Since Frog is a pure programing language using lazy evaluation is optimal because we know that a Frog function always returns the same value from the same set of arguments. We chose not to utilise eager evaluation in Frog because Frog does not support I/O and error features. Additionally, using a lazy evaluation approach makes it possible, as previously mentioned, to add features like streams.

\para
Furthermore, we have also chosen to use a lazy evaluation approach in OTTR templates to determine when OTTR should evaluate function calls—resulting in OTTR only evaluating a function call when the function call reaches a base template. Why lazy evaluation is a better approach than eager evaluation, in this case, is that OTTR does not force a user to use all variables in the template, even though OTTR throws a warning\lhk{as instances with \emph{none} as arguments to mandatory parameters are discarded, some arguments may actually be used but still not needed}. Therefore, using lazy evaluation assures us that OTTR only evaluates function calls needed to expand the template.  

\para
One of the benefits of using lazy evaluation is memorisation. As previously mentioned, OTTR only evaluates a function call when it reaches a base template—utilising memorisation is, therefore, an advantage in OTTR because we do not know how many base templates a function call reaches. In summation, lazy evaluation and memorisation assure that OTTR only evaluates a function call once and only if the result of the function is needed.

\subsubsection{Type sytem in Frog}
\label{types in frog}
As already stated, Frog uses the same type system as OTTR. Using the same type system is beneficial because OTTR and Frog assure us that the term returned by a function matches\lhk{perhaps rather use "is compatible with"} the type the OTTR template parameter requires. Frog and OTTR using the same type system is especially important for the OTTR specific types, such as BOT, LUB-types and list-types, because the OTTR specific types may not have an optimal counterpart in another type system. A consequence of Frog using the same type system as OTTR is that we need to append a sort of types, namely, the function type.  In the OTTR type system, there is an infinite set of function types, just like there are an infinite set of list types. The type system has an infinite set of function types because there is a new function type for every combination of parameter types and return type. 

\para
Even though OTTR and Frog use lazy evaluation, OTTR needs to check that the term returned by the function has the correct type. Since OTTR, as mentioned, throws an error if an argument has the incorrect type. Checking that the type matches\lhk{perhaps rather use "is compatible with"} is necessary to do at this stage since the base template may use a more general type. A more general type in this sense means that the type is a supertype of the original type. An excellent example is ottr:Triple, where the third parameter has the type top; meaning that the third parameter type is compatible with every term because the top type is the supertype of all types. An additional advantage of doing the type checking at this stage is that OTTR detects errors earlier. An alternative approach is to store the function call and the expected type of the argument in an object. As a result of storing the expected type and the function call in an object, OTTR can perform the type checking when the function call reaches a base template. However, this approach is not optimal since OTTR throws a typing error regardless of the function call reaching a base template. 

\para 
As mentioned, Frog makes it feasible to make anonymous functions straight into the OTTR template. Anonymous functions do not state their return type. Since OTTR needs to know the return type of a function call, Frog needs to use some typing rules to determine the return type based on the function body and the parameter type. In contrast to named functions, the anonymous functions do not explicitly state their parameter types. Frog solves this issue by setting the type of the parameters to the same type as the given argument term has\lhk{This is a bit unclear. What is meant by "argument term" here?}.

\subsubsection{Validation of functions}
At some point, Frog needs to validate the syntax and semantics of a function. For the time being, we have only discussed validating that a function's return type matches the OTTR templates corresponding parameter type. OTTR does some validations before expanding an instance, such as checking that the syntax is correct. Naturally, Frog also does some validation before OTTR starts to expand the instances. Just as OTTR, Frog also checks the syntax in advance. In addition, Frog validates some parts of the semantic. For instance, Frog validates that every function used in the function receives terms with the correct type. Imagine a scenario where we have two functions: add and x. In this scenario, add takes two integers, and x utilises the add function. Then Frog validates that the two arguments given to add are terms of type xsd:integer. Furthermore, Frog also validates that a function and a function utilised in a function returns a term of the type that they state.

\subsubsection{Termination}
As mention in \autoref{type theory}, the type system used in a programming language can guarantee termination, often defined as type-based termination \autocite{TypeTermination}. Another approach is to use a set of base function or combinators that builds up new functions \autocite{TypeTermination}, as described in \autoref{combinatory logic}. Guaranteed termination is a wanted feature to have in Frog since a valid OTTR template guarantees termination. If Frog does not guarantee termination, then the OTTR templates utilising Frog function can not ensure termination. 


\subsection{Alternative approaches}
In this section, we propose alternative approaches \lhk{perhaps add here "the problems addressed by Frog"} to Frog. We group these approaches into two collections. Firstly, we consider using different Semantic Technologies, namely SPARQL, OWL, and SWRL. Additionally, we introduce a way to solve some of the example issues we presented in \autoref{motivation} using these Semantic Technologies. Secondly, we discuss two different languages for computation on RDF, Ripple and Adenine, and how we may use these languages as alternatives to Frog.

\subsubsection{SPARQL}
SPARQL \autocite{SPARQL} is a query language over RDF with many similarities to SQL for relational databases, such as SELECT, WHERE, and GROUP BY. These clauses are built up similarly in SPARQL and SQL. Like SQL, SPARQL also offers clauses to delete and insert data. Additionally, SPARQL offers BIND, which allows us to assign values to a variable. Consequently, BIND makes it possible to make simple functions and bind the result of the function to a variable. Therefore, using a combination of an insert clause and BIND makes it possible to add calculated data to the graph.

\para
In \autoref{motivation}, we described an example where we wanted to store degrees in Celcius. However, some of the sources used Fahrenheit, resulting in a needed conversion. A possible solution to this problem, and similar ones, is to use SPARQL. Using SPARQL, we would first make OTTR templates and instances, both for Fahrenheit and Celcius, and expand these instances. Secondly, we would make a BIND form that converts the existing Fahrenheit degrees to Celcius and use an insert clause to add these computed Celcius degrees to the graph. Lastly, we would use a delete clause to delete all the triples which contain the degrees in Fahrenheit.

\para
Using a SPARQL query on the expanded graph rather than using a FROG function in the OTTR template may yield the same results, meaning that SPARQL could be an alternative approach to FROG. However, in addition to SPARQL demanding more steps to get the same result, this technique does not preserve the benefit of abstraction, as well as FROG does. Having several steps may result in more maintenance. In addition, one of FROG goals is that we only can manipulate the terms and not the whole graph. Using SPARQL, on the other hand, will manipulate the graph and not only the term. Consequently, using SPARQL as an alternative would not preserve the benefits of using FROG.

\subsubsection{OWL}
Web Ontology Language (OWL) \autocite{OWL} is an ontology language for the Semantic Web, based on description logic. OWL provides us with classes, data properties, object properties, and instances\lhk{perhaps rather use the word "individuals" here, to not confuse it with OTTR's "instances"}. In addition, OWL defines ways to model classes, properties, and instances using logic. OWL makes it feasible to model role restrictions. Role restrictions let us model classes as a set of objects that either only have a role to a given set \lhk{there should not be a space before the footnote-command}\footnote{universal restriction in description logic} or has some role to a given set \footnote{existential restriction in description logic}.  With role descriptions and OWL, we can, for instance, express that a student takes some courses or that all plants only drink water. OWL also provides cardinality on roles, making it possible to express that all cats have four paws.

\para
Combined with reasoning, OWL can add inferred data to a graph. If we take the life stage example from the \autoref{motivation}, we can make an ontology with OWL stating that all ex-p:Adult has some age greater than 17, and all ex-p:Child has some age less than 18—giving the persons the correct type.

\para 
To conclude, OWL can add inferred data to a graph. However, this is not the primary goal of Frog, but more an additional feature. Moreover, using OWL has restrictions. We could not, for instance, use OWL to model that something with age above 17 has the ex-r:lifeStage ex-p:Adult. Meaning that OWL expects the data to be structured. Frog, on the other hand, only works on terms. Additionally, making functions or performing computations on terms is impossible with OWL.

\subsubsection{SWRL}
SWRL \autocite{SWRL} is a proposed language for the Semantic Web. SWRL can express rules and logic by combining subsets of OWL and the Rule Markup language. In addition, SWRL also offers several statements that work like functions, which we can use on different types. However, SWRL does not allow us to make new statements\lhk{I guess you mean "terms" here, as the point of SWRL is to infer/add new statements} only to use and combine predefined statements. Furthermore, an SWRL statement has two parts: a body and a head. The body is a set of conditions, while the head is the consequence of these conditions.

\para
Moreover, we can use SWRL to solve the life stage example and the Fahrenheit to Celcius example and similar problems. First, in the life stage example, we need to make two rules: one rule for the adult and one for the child. The first conditions are equal for both rules, namely having an age. In addition, the rule that ends up with the consequence rdf:type being ex-p:Child needs a condition stating that the age is below 18, using the swrl:lessThan statement \autocite{SWRL}.  On the other hand, concluding that the rdf:type is ex-p:Adult, needs a second condition declaring that the age is greater than 17, using swrl:greaterThan \autocite{SWRL}. Likewise, SWRL can convert Fahrenheit to Celcius, where we in the consequence can combine several math functions offered by SWRL to convert these degrees.\lhk{Perhaps also write out the rules?}

\para
SWRL yields the same result as Frog in many cases. Since SWRL does not have as strict rules as OWL, SWRL also let us express, for instance, that something with age over 17 has the ex-r:lifeStage ex-p:Adult. However, using SWRL as an alternative to Frog would not be optimal. Firstly, using a rule language, like SWRL, means that we have to do reasoning after expanding the instances resulting in it being impossible for SWRL to perform a calculation that OTTR needs during the expansion, such as creating URIs. Secondly, SWRL only has a closed set of base functions or built-ins, and these base function mainly works on different xsd types and lists. However, OTTR needs Frog to be able to perform manipulation on terms of any type.

\subsubsection{Ripple}
Ripple is a functional, stack-based query language that allows us to make queries on RDF graphs and functions on RDF terms. Ripple's interface has two different statements: making queries and making commands or directives. Combining the command, define, and querying can make functions. One of the results of Ripple being a stack-based language is that the evaluation strategy mainly focuses on the separation of active and passive \footnote{active stack items exhibit a type of operation on a specific number of arguments, while passive stack items push them-self to the stack} stack items.\autocite{ShinavierJoshua_FPLD}.

\para
Ripple has numerous similarities to Frog, such as being functional and using a nested list structure. Compared to Semantic Technologies, making a function call using a Ripple function inside an OTTR template may be possible. As a result, making necessary calculations before expanding the instances, such as making a URI, may be possible. However, using Ripple as an alternative may yield one problem: a difference in typing and term systems. Meaning that, for instance, checking that a return type matches the expected type of the argument may be difficult. These problems may be complicated to solve for the OTTR specific types like the LUB and list types.  As mentioned in \autoref{frog_evaluation}, we only want OTTR to evaluate a term once and only if it is needed. However, to the best of our knowledge, Ripple does not offer memorization. Therefore, applying Ripple may result in evaluating the same expression several times.   

\subsubsection{Adenine}
Another alternative approach that is similar to Ripper\lhk{Ripple?} is Adenine. Adenine is a language written for project Haystack to more effectively make and manipulate RDF-data, meaning that Adenine can change the graph directly by, for example, adding new tripels. Adenine is an imperative language that has a syntax resembling a combination of Python and Lisp \autocite{Adenine}.

\para
Like Ripper, one of Adenine's problems as an alternative approach to Frog is that OTTR and Adenine do not utilise the same type or term system. Additionally, Adenine does not change or manipulate terms but whole graphs. Consequently, Adenine needs, similar to Semantic Technologies, to work on the expanded OTTR instances. Working on the expanded instances means that it is impossible to calculate values that OTTR needs while expanding the instances. 

\subsubsection{Summarisation}
To summarise, we have discussed five different approaches to Frog. These approaches, except Ripper, have one common disadvantage: they can not do calculations while OTTR expands the templates. The result is that it is impossible to make terms that OTTR needs when expanding, such as making a URI. An additional reason to not use any of these four approaches is that they somehow modify the graph. However, in our case, we do not want to modify the graph, only the terms. Furthermore, both OWL and SWRL have restrictions, making it impossible to solve some cases utilising only OWL and SWRL. After these disadvantages, we have one approach left, Ripper. However, Ripper does not have the same type and term system as OTTR, which may lead to the aforementioned difficulties. Using a scripting language may lead to the same difficulties as Ripper regarding the type and term system. 

\begin{table}[ht!]
    \begin{tabular}{|c|c|c|c|c|c|}
        \hline
        & SPARQL & OWL & SWRL & Ripper & Adenine \\ \hline
        Manipulate terms during the expansion & \redx & \redx & \redx & \greencheck & \redx \\ \hline
        Does not modify the graph & \redx & \redx & \redx & \greencheck & \redx\\ \hline
        Does not have too significant restrictions  & \greencheck & \redx & \redx & \greencheck & \greencheck \\ \hline
        Type checking and verification  & \redx & \redx & \redx & \redx &\redx \\ \hline
        Accessible (or open-source)  & \greencheck & \greencheck & \greencheck & \greencheck & \redx\tablefootnote{To the best of our knowledge, there does not exist an open-source version of Adenine} \\ \hline
    \end{tabular}

\end{table}

\subsection{Summary}
In summary, we want to make a programming langue, Frog, to perform manipulation on OTTR terms. Frog should be a high-lever, pure typed language and use lazy evaluation. In addition, Frog will use the same type system as OTTR. However, we need to add function types that are affected by the parameter types and return type.  Furthermore, we will validate the Frog functions, validating the syntax and the typing. 

\begin{itemize}
    \item Utvikle et funksjonelt, highlever, typed, pure språk, frog som skal kunne modifisere OTTR termer
    \item Modification the OTTR type system with appending a new sort of types, function types.
    \item Implementing lazy evaluation to evaluate Frog functions and decide when to evaluate them in the template expansion.
    \item Implement type checking in the function and when using a function call in OTTR
    \item Implement a syntax checker for Frog
\end{itemize}
